\noindent\underline{\textbf{Potensial Listrik}}
\begin{enumerate}
    \item Suatu \textit{Geiger Counter} (alat pencacah partikel) terdiri dari logam berbentuk silinder dengan diameter $2.1cm$. Di sumbu silinder terdapat suatu kawat berdiameter $1.34\times10^{-4}cm$. Jika diantara logam dan kawat diberi tegangan $855V$, hitung besar medan listrik pada permukaan kawat dan permukaan silinder!

    \item Suatu pesawat luar angkasa bergerak melewati sekumpulan gas yang terionisasi pada ionosfer bumi. Potensial pesawat ini berubah $-1$ \textit{Volt} saat ia menempuh satu putaran. Degan menganggap pesawat berbentuk bola berjari-jari $10m$, perkirakan berapa jumlah muatan yang dikumpulkannya!

    \item Suatu tetes air membawa muatan $32.0pC$ mempunyai potensial $512V$ di permukaannya. Hitung jari-jari tetesan ini! Jika dua tetesan yang sejenis dengan jari-jari yang sama dengan tetesan di atas bergabung menjadi satu tetes besar. Hitung potensial pada tetes yang baru!

    \item Suatu gelembung sabun berjari-jari $10cm$ dengan ketebalan $3.3\times 10^{-6}cm$ diberi muatan sampai mencapai potensial sebesar $100V$. Gelembung ini kemudian pecah dan menjadi suatu tetesan berbentuk bola. Hitung potensial tetesan ini!.
\end{enumerate}
